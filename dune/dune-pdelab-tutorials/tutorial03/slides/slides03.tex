%%%%%%%%%%%%%%%%%%%%%%%%%%%%%%%%%%%%%%%%%%%%%%%%%%%%%%%%%%%%%%%%%%%%%%%%%%%%%%%%
%%%%%%%%%%%%%%%%%%%%%%%%%%%%%%%%%%%%%%%%%%%%%%%%%%%%%%%%%%%%%%%%%%%%%%%%%%%%%%%%
%
% A general frame for lecture slides and lecture notes in one file
% using LaTeX beamer
%
%%%%%%%%%%%%%%%%%%%%%%%%%%%%%%%%%%%%%%%%%%%%%%%%%%%%%%%%%%%%%%%%%%%%%%%%%%%%%%%%
%%%%%%%%%%%%%%%%%%%%%%%%%%%%%%%%%%%%%%%%%%%%%%%%%%%%%%%%%%%%%%%%%%%%%%%%%%%%%%%%
\documentclass[ignorenonframetext,11pt]{beamer}
%\usepackage[ngerman]{babel}
%\usepackage[T1]{fontenc}
\usepackage[utf8]{inputenc}
\usepackage{lmodern}
\usepackage{amsmath,amssymb,amsfonts}


% only presentation
\mode<presentation>
{
  \usetheme{default}
%  \usecolortheme{crane}
  \setbeamercovered{transparent}
%  \setlength{\parindent}{0pt}
%  \setlength{\parskip}{1.35ex plus 0.5ex minus 0.3ex}
%  \usefonttheme{structuresmallcapsserif}
  \usefonttheme{structurebold}
  \setbeamertemplate{theorems}[numbered]
  \usepackage{amscd}
}

% all after
\usepackage{tikz}
\usetikzlibrary{patterns}
\usepackage{pgfplots,adjustbox}
\usepackage{eurosym}
\usepackage{graphicx}
\usepackage{multimedia}
\usepackage{psfrag}
\usepackage{listings}
\lstset{language=C++, basicstyle=\ttfamily,
  keywordstyle=\color{black}\bfseries, tabsize=4, stringstyle=\ttfamily,
  commentstyle=\it, extendedchars=true, escapeinside={/*@}{@*/}}
\usepackage{curves}
%\usepackage{epic}
\usepackage{calc}
%\usepackage{picinpar}
%\usepackage{fancybox}
%\usepackage{xspace}
\usepackage{enumerate}
\usepackage{algpseudocode}
\usepackage{color}
\usepackage{bold-extra}
\usepackage{bm}
\usepackage{stmaryrd}
%\usepackage[squaren]{SIunits}
\usepackage{nicefrac}

\usepackage{fancyvrb,bbm,xspace}
\usepackage{lmodern}
\usepackage{fancyvrb,bbm,xspace}
\usepackage[binary-units]{siunitx}
\usepackage{xcolor,tabu}

\definecolor{niceblue}{rgb}{0.122,0.396,0.651}   %% 31, 101, 166 or #1F65A6
\definecolor{niceorange}{RGB}{255,205,86}        %% #FFCD56
\definecolor{nicered}{RGB}{220,20,60}                      %% rgb(220, 20, 60)
\definecolor{niceteal}{HTML}{00A9AB}
\definecolor{niceviolet}{HTML}{820080}

\definecolor{niceblueLight}{HTML}{91CAFB}
\definecolor{niceblueVeryLight}{HTML}{DDEFFF}

\usepackage{dsfont}

%\newcommand{\hlineabove}{\rule{0pt}{2.6ex}}
%\newcommand{\hlinebelow}{\rule[-1.2ex]{0pt}{0pt}}

%\usecolortheme[RGB={37,75,123}]{structure}
% \definecolor{structurecolor}{rgb}{0.905,0.318,0.071}

% \setbeamercolor{frametitle}{fg=black,bg=}
% \setbeamercolor{sidebar left}{fg=,bg=}

% \setbeamertemplate{headline}{\vskip4em}
% \setbeamersize{sidebar width left=.9cm}

% \setbeamertemplate{navigation symbols}{}
%\setbeamertemplate{blocks}[rounded][shadow=true]
%\setbeamertemplate{itemize items}[square]

\mode<presentation>
{
\theoremstyle{definition}
}
\newtheorem{Def}{Definition}%[section]
\newtheorem{Exm}[Def]{Example}
\newtheorem{Lem}[Def]{Lemma}
\newtheorem{Rem}[Def]{Remark}
\newtheorem{Rul}[Def]{Rule}
\newtheorem{Thm}[Def]{Theorem}
\newtheorem{Cor}[Def]{Corollary}
\newtheorem{Obs}[Def]{Observation}
\newtheorem{Ass}[Def]{Assumption}
\newtheorem{Pro}[Def]{Property}
\newtheorem{Alg}[Def]{Algorithm}
\newtheorem{Prp}[Def]{Proposition}
\newtheorem{Lst}[Def]{Listing}

% Delete this, if you do not want the table of contents to pop up at
% the beginning of each subsection:
\AtBeginSection[]
{
  \begin{frame}<beamer>
    \frametitle{Contents}
    \tableofcontents[sectionstyle=show/shaded,subsectionstyle=hide/hide/hide]
%\tableofcontents[currentsection]
  \end{frame}
}

% Title definition
\mode<presentation>
{
  \title{DUNE PDELab Tutorial 03\\
  {\small  Conforming Finite Elements for a Nonlinear Heat Equation}}
  \author{Peter Bastian}
  \institute[]
  {
   Interdisziplinäres Zentrum für Wissenschaftliches Rechnen\\
   Im Neuenheimer Feld 205, D-69120 Heidelberg \\[6pt]
  }
  \date[\today]{\today}
}


% logo nach oben
\mode<presentation>
{
% No navigation symbols and no lower logo
\setbeamertemplate{sidebar right}{}

% logo
\newsavebox{\logobox}
\sbox{\logobox}{%
    \hskip\paperwidth%
    \rlap{%
      % putting the logo should not change the vertical possition
      \vbox to 0pt{%
        \vskip-\paperheight%
        \vskip0.35cm%
        \llap{\insertlogo\hskip0.1cm}%
        % avoid overfull \vbox messages
        \vss%
      }%
    }%
}

\addtobeamertemplate{footline}{}{%
    \usebox{\logobox}%
}
}

%%%%%%%%%%%%%%%%%%%%%%%%%%%%%%%%%%%%%%%%%%%%%%%%%%%%%%%%%%%%%%%%%%%%%%%%%%%%%%%%
%%%%%%%%%%%%%%%%%%%%%%%%%%%%%%%%%%%%%%%%%%%%%%%%%%%%%%%%%%%%%%%%%%%%%%%%%%%%%%%%
%
% now comes the individual stuff lecture by lecture
%
%%%%%%%%%%%%%%%%%%%%%%%%%%%%%%%%%%%%%%%%%%%%%%%%%%%%%%%%%%%%%%%%%%%%%%%%%%%%%%%%
%%%%%%%%%%%%%%%%%%%%%%%%%%%%%%%%%%%%%%%%%%%%%%%%%%%%%%%%%%%%%%%%%%%%%%%%%%%%%%%%

\begin{document}

\frame{\titlepage}

%%%%%%%%%%%%%%%%%%%%%%%%%%%%%%%%%%%%%%%%%%%%%%%%%%%%%%%%%%%%%%%%%%%%%%%%%%%%%%%%
%%%%%%%%%%%%%%%%%%%%%%%%%%%%%%%%%%%%%%%%%%%%%%%%%%%%%%%%%%%%%%%%%%%%%%%%%%%%%%%%

\begin{frame}
\frametitle{PDE Problem}
This tutorial extends the problem from tutorial 01 to the instationary setting:
\begin{align*}
\partial_t u -\Delta u + q(u) &= f &&\text{in $\Omega\times\Sigma$},\\
u &= g &&\text{on $\Gamma_D\subseteq\partial\Omega$},\\
-\nabla u\cdot \nu &= j &&\text{on $\Gamma_N=\partial\Omega\setminus\Gamma_D$},\\
u &= u_0 &&\text{at $t=0$}.
\end{align*}
\textit{Weak formulation:} Find $u\in L_2(t_0,t_0+T;u_g+V(t))$:
\begin{equation*}
\frac{d}{dt} \int_\Omega u v \,dx+ \int_\Omega \nabla u \cdot \nabla v
+ q(u) v - f v \, dx + \int_{\Gamma_N} jv \, ds = 0 \qquad
\begin{array}{l}
\forall v \in V(t),\\
t \in \Sigma,
\end{array}
\end{equation*}
where $V(t) = \{v\in H^1(\Omega)\,:\, \text{$v=0$ on $\Gamma_D(t)$}\}$
and $H^1(\Omega)\ni u_g(t)|_{\Gamma_D}=g$.
\end{frame}

\begin{frame}
\frametitle{Residual Forms}
\textit{Introduce the residual forms:} Find $u\in L_2(t_0,t_0+T;u_g+V(t))$:
\begin{equation*}
\frac{d}{dt} m^{\text{L2}}(u,v) + r^{\text{NLP}}(u,v) = 0 \quad \forall v \in V(t), t \in \Sigma.
\end{equation*}
with the \textit{temporal residual form}
\begin{equation*}
m^{\text{L2}}(u,v) = \int_\Omega u v \,dx
\end{equation*}
and the \textit{spatial residual form}
\begin{equation*}
r^{\text{NLP}}(u,v) = \int_\Omega \nabla u \cdot \nabla v + (q(u)-f)v\,dx + \int_{\Gamma_N} jv\,ds ,
\end{equation*}
\end{frame}

\begin{frame}
\frametitle{Method of Lines}
\begin{enumerate}[1)]
\item Discretize in space with conforming finite elements, i.e. choose a
finite-dimensional test space $V_h(t)\subset V(t)$.
\item Results in a system of ordinary differential equations (ODEs) for the time-dependent
coefficient vector $z(t)$
\item Choose an appropriate method to integrate the ODE system
\item Other schemes can be implemented in PDELab as well, e.g. space-time methods
\end{enumerate}
\end{frame}

\begin{frame}
\frametitle{One Step $\theta$ Method}
Subdivide time interval
\begin{equation*}
\overline{\Sigma} = \{t^{0}\} \cup (t^0,t^1] \cup \ldots \cup (t^{N-1},t^N]
\end{equation*}
Set  $\Delta t^k=t^{k+1}-t^k$;
Find $u_h^{k+1}\in U_h(t^{k+1})$ s.t.:
\begin{equation*}
\begin{split}
\frac{1}{\Delta t_k}(&m_h^\text{L2}(u_h^{k+1},v;t^{k+1})-m_h^\text{L2}(u_h^{k},v;t^{k})) + \\
&\theta r_h^\text{NLP}(u_h^{k+1},v;t^{k+1}) + (1-\theta) r_h^\text{NLP}(u_h^{k},v;t^{k}) = 0
\quad \forall v\in V_h(t^{k+1})
\end{split}
\end{equation*}
\hrule
\smallskip
Reformulated this formally corresponds in each time step to the nonlinear problem
\begin{equation*}
\text{Find $u_h^{k+1}\in U_h(t^{k+1})$ s.t.:}
\quad r_h^{\theta,k} (u_h^{k+1},v) + s_h^{\theta,k}(v) = 0
\quad \forall v\in V_h(t^{k+1})
\end{equation*}
where
\begin{align*}
r^{\theta,k}_h(u,v) &= m_h^\text{L2}(u,v;t^{k+1})+\Delta t^k \theta r_h^\text{NLP}(u,v;t^{k+1}) ,\\
s^{\theta,k}_h(v) &= -m_h^\text{L2}(u^k_h,v;t^k) + \Delta t^k (1-\theta) r_h^\text{NLP}(u_h^k,v;t^k)
\end{align*}
\end{frame}

\begin{frame}
\frametitle{Runge-Kutta Methods}
in Shu-Osher form:
\begin{enumerate}
\item $u_h^{(0)} = u_h^{k}$.
\item For $i=1,\ldots,s\in\mathbb{N}$, find $u_h^{(i)}\in u_{h,g}(t^k+d_i \Delta t^k)
+ V_h(t^{k+1})$:
\begin{equation*}
\begin{split}
\sum\limits_{j=0}^{s} \bigl[a_{ij} m_h&\left(u_h^{(j)},v;t^k+d_j \Delta t^k\right) \\
&+ b_{ij} \Delta t^k r_h\left(u_h^{(j)}, v;t^k+d_j \Delta t^k\right) \bigr] = 0
\qquad \forall v\in V_h(t^{k+1})
\end{split}
\end{equation*}
\item $u_h^{k+1} = u_h^{(s)}$.
\end{enumerate}
An $s$-stage scheme is given by the parameters
\begin{equation*}
A = \left[\begin{array}{ccc}
a_{10} & \ldots & a_{1s}\\
\vdots &  & \vdots\\
a_{s0} & \ldots & a_{ss}
\end{array}\right],
\quad B = \left[\begin{array}{ccc}
b_{10} & \ldots & b_{1s}\\
\vdots &  & \vdots\\
b_{s0} & \ldots & b_{ss}
\end{array}\right],
\quad d = \left(
d_{0}, \ldots, d_{s}
\right)^T
\end{equation*}
Consider explicit and diagonally implicit schemes
\end{frame}

\begin{frame}
\frametitle{Examples}
\begin{itemize}
\item One step $\theta$ scheme (introduced above):
\begin{equation*}
A = \left[\begin{array}{cc}
-1 & 1
\end{array}\right],
\quad B = \left[\begin{array}{cc}
1-\theta & \theta
\end{array}\right],
\quad d = \left(
0, 1
\right)^T.
\end{equation*}
Explicit/implicit Euler ($\theta\in\{0,1\}$), Crank-Nicolson ($\theta=\nicefrac12$).
\item Heun's second order explicit method
\begin{equation*}
A = \left[\begin{array}{ccc}
-1 & 1 & 0\\
-\nicefrac12 & -\nicefrac12 & 1\\
\end{array}\right],
\quad B = \left[\begin{array}{ccc}
1 & 0 & 0\\
0 & \nicefrac12 & 0\\
\end{array}\right],
\quad d = \left(
0, 1, 1
\right)^T.
\end{equation*}
\item Alexander's two-stage second order strongly S-stable scheme
{\footnotesize\begin{equation*}
A = \left[\begin{array}{ccc}
-1 & 1 & 0\\
-1 & 0 & 1\\
\end{array}\right],
\quad B = \left[\begin{array}{ccc}
0 & \alpha     & 0\\
0 & 1-\alpha & \alpha\\
\end{array}\right],
\quad d = \left(
0, \alpha, 1
\right)^T
\end{equation*}}
with $\alpha=1-\nicefrac{\sqrt{2}}{2}$.
\item Fractional step $\theta$, three stage second order strongly A-stable
{\footnotesize\begin{equation*}
A = \left[\begin{array}{rrrr}
-1 & 1 & 0 & 0\\
0  & -1 & 1 & 0\\
0  & 0 & -1 & 1\\
\end{array}\right],
\quad B = \left[\begin{array}{rrrr}
\theta \theta' & 2\theta^2 & 0 & 0\\
0 & 2\theta\theta' & 2\theta^2 & 0\\
0 & 0 & \theta\theta' & 2\theta^2
\end{array}\right],
\quad d = \left(
0, \theta, 1-\theta, 1
\right)^T
\end{equation*}}
with $\theta=1-\nicefrac{\sqrt{2}}{2}$, $\theta' = 1-2\theta = \sqrt{2}-1$.
\end{itemize}
\end{frame}

\begin{frame}
\frametitle{Note on Explicit Schemes}
Example: Explicit Euler method ($\theta=0$)
results in: Find $u_h^{k+1}\in U_h(t^{k+1})$ s.t.:
\begin{equation*}
 m_h^\text{L2}(u_h^{k+1},v;t)-m_h^\text{L2}(u_h^{k},v;t) +
\Delta t^k r_h^\text{NLP}(u_h^{k},v;t) = 0
\quad \forall v\in V_h(t^{k+1})
\end{equation*}
Appropriate spatial discretization results in diagonal mass matrix:
\begin{equation*}
Dz^{k+1} = s^k - \Delta t^k q^k.
\end{equation*}
Requires stability condition for $\Delta t^k$.\\
\smallskip
Use follwing algorithm:
\begin{enumerate}[i)]
\item While traversing the mesh assemble the vectors $s^k$ and
$q^k$ separately and compute the maximum time step $\Delta t^k$.
\item Form the right hand side $b^k=s^k - \Delta t^k q^k$ and ``solve'' the
diagonal system $Dz^{k+1} = b^k$ (can be done in one step).
\end{enumerate}
Extends to strong stability preserving Runge-Kutta methods
\end{frame}

\begin{frame}
\frametitle{Realization in PDELab}
\begin{enumerate}[1)]
\item The ini-file
\lstinline{tutorial03.ini} holds parameters
controlling the execution.
\item Main file \lstinline{tutorial03.cc} includes the necessary C++,
DUNE and PDELab header files;
contains \lstinline{main} function;
instantiates DUNE grid objects and calls the \lstinline{driver} function
\item Function \lstinline{driver} in file \lstinline{driver.hh} instantiates
the necessary PDELab classes and finally solves the problem.
\item File \lstinline{nonlinearheatfem.hh} contains the local operator classes
\lstinline{NonlinearHeatFEM} and \lstinline{L2} realizing the spatial
and temporal residual forms.
\item File \lstinline{problem.hh} contains a parameter class which
encapsulates the user-definable part of the PDE problem.
\end{enumerate}
\end{frame}

\end{document}
