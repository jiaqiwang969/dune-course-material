\documentclass[12pt,a4paper]{article}

\usepackage[utf8]{inputenc}
\usepackage[a4paper,total={150mm,240mm}]{geometry}
\usepackage[american]{babel}

\usepackage{float}
\usepackage{babel}
\usepackage{amsmath}
\usepackage{tikz}
\usepackage{graphicx}
\usepackage{amssymb}

\usepackage{listings}
\definecolor{listingbg}{gray}{0.95}
\lstset{language=C++,basicstyle=\ttfamily\small,frame=single,backgroundcolor=\color{listingbg}}
% \lstset{language=C++, basicstyle=\ttfamily,
%   keywordstyle=\color{black}\bfseries, tabsize=4, stringstyle=\ttfamily,
%   commentstyle=\it, extendedchars=true, escapeinside={/*@}{@*/}}

\usepackage{exercise}

\title{\textbf{Exercises for the introduction to the Grid Interface}}
\exerciselabel{Exercise}

\begin{document}

\exerciseheader

\begin{Exercise}{Constructing Dune grids}

In this exercise you will experiment with constructing different grids. The working directory
of the exercise is the same as in the previous exercise on the grid interface.

The file \lstinline!grid-exercise2.cc! contains a code, which
\begin{itemize}
 \item defines the \lstinline!GridType! to one of a list of available Dune grids,
 \item constructs a grid through one of the three basic factory concepts:
  \begin{itemize}
   \item \lstinline!StructuredgridFactory! for equidistant grids,
   \item \lstinline!GmshReader! for unstructured grids,
   \item \lstinline!TensorGridFactory! for tensor product grids,
  \end{itemize}
 \item potentially refines the grid once globally (disabled by default),
 \item fills a data structure that maps each cell to its index in the index set,
 \item Outputs this data structure to a vtk file which can be visualized in \lstinline!paraview!.
\end{itemize}

Try to construct as many different grids as possible and look at the result in paraview.
Here are some questions to guide your exploration of grid construction:
\begin{itemize}
 \item Find out (visually) how the elements in a \lstinline!YaspGrid! are ordered in the index set.
 \item Construct a structured grid with an unstructured grid manager.
 \item Load an unstructured grid from one of the \lstinline!.msh! files you find in the exercise directory.
 \item Construct a \lstinline!YaspGrid! for the domain $[-1,1]^2$.
 \item Enable the global refinement in the code and observe the effect on the index set for structured and unstructured grids.
 \item Build a tensor product \lstinline!YaspGrid! with and without global refinement. What do you observe?
\end{itemize}

\end{Exercise}

\begin{Exercise}{Visualization with ParaView}
 The purpose of this exercise is to explore the visualization capabilities of ParaView.

 We start by visualizing the file \lstinline!2dexample.vtu!. It contains two data sets: a solution of a finite element method
 for the Poisson equation and the interpolation of the exact solution onto the grid. Try the following steps:
 \begin{itemize}
  \item Load the file into paraview and get it visualized.
  \item Get a 3D Plot by using the \lstinline!Warp by Scalar! filter.
  \item Plot the solution along a line using the \lstinline!Plot over Line! filter.
  \item Add a third data set using the \lstinline!Calculator! filter, which contains the error of the finite element discretization.
 \end{itemize}

 We will now visualize the solution of the finite volume example from this morning. If you have not successfully implemented the scheme,
 you can produce the results by running the executables from the solution subfolder. Instead of a single file, we now load a \lstinline!pvd!
 file that describes a time series. Try the following:
 \begin{itemize}
  \item Load the \lstinline!pvd! file, get it visualized and run it.
  \item Use \lstinline!Plot over Line!.
  \item Try using the filter \lstinline!Warp by Scalar!. What might be the problem with it?
  \item Calculate the total mass by using the filter \lstinline!Integrate Variables!.
  \item Try plotting the total mass over time and hopefully observe mass conversation. To do this,
  you need to select the row with your integrated variable by clicking it and then apply the filter
  \lstinline!Plot Selection over Time!.
 \end{itemize}

 Now load the 3D Finite Volume Data and try to explore it with ParaView. In addition to the tools mentioned above,
 you might want to play around with \lstinline!Clip!, \lstinline!Contour! and \lstinline!Threshold!.
\end{Exercise}


\end{document}
