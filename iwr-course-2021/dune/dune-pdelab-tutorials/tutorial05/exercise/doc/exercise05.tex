\documentclass[12pt,a4paper]{article}

\usepackage[utf8]{inputenc}
\usepackage[a4paper,total={150mm,240mm}]{geometry}
\usepackage[american]{babel}

\usepackage{float}
\usepackage{babel}
\usepackage{amsmath}
\usepackage{tikz}
\usepackage{graphicx}
\usepackage{amssymb}
\usepackage{todonotes}


\usepackage{listings}
\definecolor{listingbg}{gray}{0.95}
\lstset{language=C++,basicstyle=\ttfamily\small,frame=single,backgroundcolor=\color{listingbg}}
% \lstset{language=C++, basicstyle=\ttfamily,
%   keywordstyle=\color{black}\bfseries, tabsize=4, stringstyle=\ttfamily,
%   commentstyle=\it, extendedchars=true, escapeinside={/*@}{@*/}}


\newcommand{\vx}{\vec x}
\newcommand{\grad}{\vec \nabla}
\newcommand{\wind}{\vec \beta}
\newcommand{\Laplace}{\Delta}
\newcommand{\mycomment}[1]{}


% Exercise stylesheet
\usepackage{exercise}

\title{\textbf{Exercises for Tutorial05}}
\subtitle{Adaptivity}
\exerciselabel{Exercise}


\begin{document}

\exerciseheader

\begin{Exercise}{Playing with the Parameters}
  The adaptive finite element algorithm is controlled by several
  parameters which can be set in the ini-file.  Besides the polynomial
  degree these are:
  \begin{itemize}
  \item \lstinline{steps} : how many times the solve-- adapt cycle is
    executed.
  \item \lstinline{uniformlevel} : how many times uniform mesh
    refinement is used before adaptive refinement starts.
  \item \lstinline{tol} : tolerance where the adaptive algorithm is
    stopped.
  \item \lstinline{fraction} : Remember that the global error is
    estimated from element-wise contributions:
    $$\gamma^2(u_h) = \sum_{T\in \mathcal{T}_h} \gamma_T^2(u_h).$$ The
    fraction parameter controls which elements are marked for
    refinement. More precisely, the largest threshold $\gamma^\ast$ is
    determined such that
    $$\sum_{\{T\in\mathcal{T}_h \,:\, \gamma_T(u_h)\geq\gamma^\ast\}}
    \gamma_T^2(u_h) \geq \text{\lstinline{fraction}} \cdot
    \gamma^2(u_h). $$

    If \lstinline{fraction} is set to zero no elements need to be
    refined, i.e. $\gamma^\ast=\infty$.  When \lstinline{fraction} is
    set to one then all elements need to be refined,
    i.e. $\gamma^\ast=0$.  A smaller \lstinline{fraction} parameter
    results in a more optimized mesh but needs more steps and thus
    might be more expensive. If \lstinline{fraction} is chosen too
    large then almost all elements are refined which might also be
    expensive, so there exists an (unknown) optimal value for
    \lstinline{fraction}.
  \end{itemize}
  Carry out the following experiments:
  \begin{enumerate}
  \item Choose polynomial degree 1 and fix a tolerance,
    e.g. \lstinline{tol}=0.01.  Set \lstinline{steps} to a large
    number so that the algorithm is not stopped before the tolerance
    is reached. Set \lstinline{uniformlevel} to zero. Now vary the
    \lstinline{fraction} parameter and measure the execution time with
    \begin{lstlisting}[basicstyle=\ttfamily\small,
      frame=single,
      backgroundcolor=\color{listingbg}]
$ time ./exercise05
    \end{lstlisting}%$
  \item Repeat the previous experiment with polynomial degrees two and
    three. Compare the optimal meshes generated for the different
    polynomial degrees.
  \item The idea of the heuristic refinement algorithm refining only
    the elements with the largest contribution to the error is to
    \textit{equilibrate} the error. To check how well this is achieved
    use the calculator filter in ParaView (use cell data) to plot
    $\log(\gamma_T(u_h))$. Note that the cell data exported by the
    (unchanged) code is $\gamma_T^2(u_h)$! Now check the minimum and
    maximum error contributions. What happens directly at the
    singularity, i.e. the point $(0,0)$?
  \end{enumerate}
\end{Exercise}


\begin{Exercise}{Compute the True $L_2$-error}
  \textit{Disclaimer: In this exercise we compute and evalute the
    $L_2$-norm of the error. The residual based error estimator
    implemented in the code, however, estimates the $H^1$-seminorm
    (which is equivalent to the $H^1$-norm here) and thus also
    optimizes the mesh with respect to that norm.  This is done to
    make the exercise as simple as possible. It would be more
    appropriate to carry out the experiments either with the true
    $H^1$-norm or to change the estimator to the $L_2$-norm.}

  To do this exercise we need some more background on PDELab grid
  functions.  The following code known from several tutorials by now
  constructs a PDELab grid function object \lstinline{g}:
  \begin{lstlisting}[basicstyle=\ttfamily\small,
    frame=single,
    backgroundcolor=\color{listingbg}]
Problem<RF> problem(eta);
auto glambda =
[&](const auto& e, const auto& x){return problem.g(e,x);};
auto g = Dune::PDELab::makeGridFunctionFromCallable(gv,glambda);
    \end{lstlisting}
    Also the following code segment used in many tutorials constructs
    a PDELab grid function object \lstinline{zdgf} from a grid
    function space and a coefficient vector:
    \begin{lstlisting}[basicstyle=\ttfamily\small,
frame=single,
backgroundcolor=\color{listingbg}]
typedef Dune::PDELab::DiscreteGridFunction<GFS,Z> ZDGF;
ZDGF zdgf(gfs,z);
    \end{lstlisting}
    PDELab grid functions can be evaluated in local coordinates given
    by an element and a local coordinate within the corresponding
    reference element.  The result is stored in an additional argument
    passed by reference.  Here is a code segment doing the job:
    \begin{lstlisting}[basicstyle=\ttfamily\small,
frame=single,
backgroundcolor=\color{listingbg}]
Dune::FieldVector<RF,1> truesolution(0.0);
g.evaluate(e,x,truesolution);
    \end{lstlisting}
    Here we assume that \lstinline{e} is an element and \lstinline{x}
    is a local coordinate.  The result is stored in a
    \lstinline{Dune::FieldVector} with one component of type
    \lstinline{RF}. This allows also to return vector-valued results.

    Now here are your tasks:
    \begin{enumerate}
    \item Extend the code in the file \lstinline{driver.hh} in
      \lstinline{tutorial05/exercise/task} to provide a grid function
      that provides the error $u-u_h$. Note that the method
      \lstinline{g} in the parameter class already returns the true
      solution (for $\eta=0$).  You can solve this task by writing a
      lambda function subtracting the values returned by
      \lstinline{g.evaluate} and \lstinline{zdgf.evaluate}.

    \item Compute the $L_2$-norm of the error, i.e.
      $\|u-u_h\|_0 = \sqrt{\int_\Omega (u(x)-u_h(x))^2\,dx}$. This can
      be done by creating a grid function with the squared error using
      \lstinline{Dune::PDELab::SqrGridFunctionAdapter} from
      \lstinline{<dune/pdelab/function/sqr.hh>} and the function
      \lstinline{Dune::PDELab::integrateGridFunction} contained in the
      header file
      \lstinline{<dune/pdelab/common/functionutilities.hh>}.  Output
      the result to the console by writing in a single line a tag like
      \lstinline{L2ERROR}, the number of degrees of freedom using
      \lstinline{gfs.globalSize()} and the $L_2$-norm. This lets you
      grep the results later for post-processing.

    \item Investigate the $L_2$-error for different polynomial
      degrees. Run the code using different polynomial degrees and
      also uniform and adaptive (use your optimal
      \lstinline{fraction}) refinement. Use \lstinline{gnuplot} to
      produce graphs such as those in Figure \ref{Fig:L2}. An example
      gnuplot file \lstinline{plot.gp} is given.

      The graph shows that for uniform refinement polynomial degree 2
      is better than polynomial degree 1 only by a constant
      factor. For adaptive refinement a better convergence rate can be
      achieved. The plot also shows that for adaptive refinement the
      optimal convergence rate can be recovered. For $P_1$ finite
      elements and fully regular solution in $H^2(\Omega)$ the
      a-priori estimate for uniform refinement yields
      $\|u-u_h\|_0\leq C h^2$. Expressing $h$ in terms of the number
      of degrees of freedom $N$ yields $h\sim N^{-1/d}$, so
      $\|u-u_h\|_0\leq C h^2 \leq C' N^{-1}$ for $d=2$.  The curve
      $N^{-1}$ is shown for comparison in the plot. Clearly, the
      result for $P_1$ is parallel to that line.
    \item Optional: Also write the true error to the VTK output file.
    \end{enumerate}
\end{Exercise}


\begin{figure}
\begin{center}
\includegraphics[width=\textwidth]{l2error}
\end{center}
\caption{$L_2$-error versus number of degrees of freedom for different polynomial degrees.}
\label{Fig:L2}
\end{figure}

\end{document}
