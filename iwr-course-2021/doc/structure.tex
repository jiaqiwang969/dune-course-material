\documentclass{article}
\usepackage[margin=.75in]{geometry}
\usepackage{dirtree}
\usepackage{wrapfig}

\parskip1ex
\setlength{\parindent}{0ex}

\begin{document}

\par\bigskip
  \begin{center}
    \large{Structure of the course material} \par
    for the IWR Dune course
  \end{center}
  \rule{\textwidth}{1pt}
  \bigskip

\begin{wrapfigure}{l}{6cm}
\dirtree{%
.1 course.
.2 dune.
.3 dune-common.
.3 dune-grid.
.3 {...}.
.3 dune-pdelab-tutorials.
.4 tutorial00.
.5 src.
.5 doc.
.5 exercise.
.6 task.
.6 doc.
.6 solution.
.4 tutorial01.
.5 {...}.
.4 {...}.
%.2 external.
%.3 ug.
.2 release-build.
.3 dune-common.
.3 dune-grid.
.3 {...}.
.3 dune-pdelab-tutorials.
.4 src\_dir.
.4 tutorial00.
.5 src\_dir.
.5 src.
.6 src\_dir.
.5 doc.
.6 src\_dir.
.5 exercise.
.6 src\_dir.
.6 task.
.7 src\_dir.
.6 doc.
.7 src\_dir.
.6 solution.
.7 src\_dir.
.4 tutorial01.
.5 {...}.
.4 {...}.
%.2 debug-build.
%.3 {...}.
.2 install.sh.
.2 external.sh.
.2 update.sh.
%.2 debug.opts.
.2 release.opts.
}
\end{wrapfigure}

The course material for the dune-pdelab course at IWR is distributed in form of
a virtual machine that is running a Fedora system as the guest system. The
username is \verb!courseuser!, the password \verb!dunecourse!. In this virtual
machine, you have a folder \verb!course! in your home directory, which has the
structure shown to the left.

The \verb!dune! subfolder contains the sources of all the Dune modules
needed for the course. The course examples are located in the module
\verb!dune-pdelab-tutorials!.
% External libraries, which are interfaced by Dune, but not part of
% Dune, are built into the \verb!external!  subdirectory.

Dune uses a \verb!CMake! based build system. In \verb!CMake!, there is
a clean separation between the \textit{source directory}, which
usually is under version control (here: the \verb!dune! subdirectory)
and the \textit{build directory}, where built executables, program
output and such are placed. In this course the build directory is
\verb!release-build! where all programs are compiled with all optimizations.

The build directory mirrors the structure of the source
directory. You should by default navigate to the \verb!release-build!
subdirectory of the current exercise and work there. If you need to
access the sources of a given example, you can follow the symlink
\verb!src_dir! in any build subdirectory to switch to the
corresponding subdirectory of the source tree.

As already mentioned the course examples are in the
\verb!dune-pdelab-tutorials! module. Each subdirectory of this module
corresponds to one tutorial (\verb!tutorialxy!). They all have the
same internal structure: The \verb!src! subdirectory contains the
example code which was shown in the lecture. The \verb!doc!
subdirectory contains the latex sources of a detailed explanation of
the tutorial. The \verb!exercise! subdirectory, which is relevant for
this course, is subdivided even further: \verb!task!  contains the code
skeleton to work on during the exercise, \verb!doc! contains the
sources of the exercise sheet and \verb!solution!  contains what you
expect it to.

The shell scripts in \verb!course! set up the course
material. This has already been done, so please do not call any of
these scripts unless the course team tells you to. The two \verb!opts!
files do configure the two separate builds, there is also no need to
touch these.

In case you are not familiar with UNIX system in general, here is a
small cheat sheet of commands that we think are necessary for the
exercises:
\begin{itemize}
\item \verb!cd <dir>! \verb!c!hanges the current working
  \verb!d!irectory to \verb!dir! (to the home directory, if omitted).
\item \verb!ls! \verb!l!i\verb!s!ts the contents of the current
  working directory
\item \verb!pwd! \verb!p!rints the current \verb!w!orking
  \verb!d!irectory
\item \verb!g++ <options> <sources>! compiles C++ sources (only needed
  in the C++ exercise)
\item \verb!make <executablename>! (re)builds executables in the
  current build directory. If the executable name is omitted all
  executables in the current directory are built.
 \item \verb!paraview! is a visualization program for VTK files
\end{itemize}

\end{document}

